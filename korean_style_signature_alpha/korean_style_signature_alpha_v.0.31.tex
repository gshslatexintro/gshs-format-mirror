% 2017/01/11 v.0.31
% (인) 표시 위에 서명/날인 그림을 넣는 방법. 알파 버전.
% v.0.2 양식을 가져와서 편집하였음.
% 해결됨 : % 서명의 종횡비가 클 경우 심사위원 이름을 침범.
% http://tex.stackexchange.com/questions/89588/positioning-relative-to-page-in-tikz
% 위 웹사이트에 나온 방법을 통해 tikz로 페이지에 대한 absolute position 에 tikzpicture를 넣는 방법을 사용하였음.
% 절대 위치를 사용하기 때문에 논문 제목의 줄 수에 따라 위치가 변동될 우려 또한 없음.

\documentclass[a4paper]{article}
\usepackage[showframe,top=35mm,bottom=30mm,left=30mm,right=30mm]{geometry}
% 여백을 보여주기 위해 showframe 옵션 사용.
\usepackage{graphicx}
\usepackage{tikz} % tikzpicture를 사용하기 위함.
\usepackage{kotex}
\usepackage{adjustbox} % trimbox를 사용하기 위함.
\usepackage{calc} %길이 계산을 위함.
\usepackage{ifthen}


\newenvironment{myepigraph}
{\par\hfill
	\begin{tabular}{@{}r@{\hspace{20mm}}}%
	}%
	{%
	\end{tabular}\par\medskip}

\def\apprvsignone{parkdaegam.png}	%첫 번째 서명 파일
\def\apprvsigntwo{ntiea.png}		%두 번째 서명 파일
\def\apprvsignthree{hongpanseo.png}	%세 번째 서명 파일
\def\apprvsign{(인)}

\newlength{\apprvsignonewidth}
\newlength{\apprvsigntwowidth}
\newlength{\apprvsignthreewidth}
\newlength{\apprvsignoneheight}
\newlength{\apprvsigntwoheight}
\newlength{\apprvsignthreeheight}
\settowidth\apprvsignonewidth{\includegraphics[height=12mm]{\apprvsignone}}
\settowidth\apprvsigntwowidth{\includegraphics[height=12mm]{\apprvsigntwo}}
\settowidth\apprvsignthreewidth{\includegraphics[height=12mm]{\apprvsignthree}}
\setlength{\apprvsignoneheight}{12mm}
\setlength{\apprvsigntwoheight}{12mm}
\setlength{\apprvsignthreeheight}{12mm}
\newcommand{\apprvsignonevalid}{true}
\newcommand{\apprvsigntwovalid}{true}
\newcommand{\apprvsignthreevalid}{true}
\ifthenelse{\lengthtest{\apprvsignonewidth < 25mm}}{}{%
	\renewcommand{\apprvsignonevalid}{false}%
	\setlength{\apprvsignonewidth}{25mm}%
	\settoheight\apprvsignoneheight{%
		\includegraphics[width=25mm]{\apprvsignone}%
	}
}
\ifthenelse{\lengthtest{\apprvsigntwowidth < 25mm}}{}{%
	\renewcommand{\apprvsigntwovalid}{false}%
	\setlength{\apprvsigntwowidth}{25mm}%
	\settoheight\apprvsigntwoheight{%
		\includegraphics[width=25mm]{\apprvsigntwo}%
	}
}
\ifthenelse{\lengthtest{\apprvsignthreewidth < 25mm}}{}{%
	\renewcommand{\apprvsignthreevalid}{false}%
	\setlength{\apprvsignthreewidth}{25mm}%
	\settoheight\apprvsignthreeheight{%
		\includegraphics[width=25mm]{\apprvsignthree}%
	}
}
\newlength{\apprvsignwidth}
\settowidth\apprvsignwidth{\apprvsign}


\makeatletter
\def\parsecomma#1,#2\endparsecomma{\def\page@x{#1}\def\page@y{#2}}
\tikzdeclarecoordinatesystem{page}{
	\parsecomma#1\endparsecomma
	\pgfpointanchor{current page}{north east}
	% Save the upper right corner
	\pgf@xc=\pgf@x%
	\pgf@yc=\pgf@y%
	% save the lower left corner
	\pgfpointanchor{current page}{south west}
	\pgf@xb=\pgf@x%
	\pgf@yb=\pgf@y%
	% Transform to the correct placement
	\pgfmathparse{(\pgf@xc-\pgf@xb)/2.*\page@x+(\pgf@xc+\pgf@xb)/2.}
	\expandafter\pgf@x\expandafter=\pgfmathresult pt
	\pgfmathparse{(\pgf@yc-\pgf@yb)/2.*\page@y+(\pgf@yc+\pgf@yb)/2.}
	\expandafter\pgf@y\expandafter=\pgfmathresult pt
}
\makeatother

\begin{document}
	\vspace*{3em}
	
	\vspace{3em}

	\begin{tikzpicture}[remember picture, overlay]
	\node[anchor=east] at (page cs:0.535,0.35+31mm){%
		%\pgfimage{example-image-a}
		{\Large {몇년몇월며칠}}
	};
	
	\node[anchor=center] at %
	(page cs:0.48+\apprvsignonewidth/2-\apprvsignwidth-2mm,0.35+14mm){%
		%\pgfimage{example-image-a}
		\includegraphics[height=\apprvsignoneheight]{\apprvsignone}
	};
	\node[anchor=east] at (page cs:0.535,0.35+14mm){%
		%\pgfimage{example-image-a}
		{\Large 심사위원장\hspace{5mm}박 대 감\hspace{6mm}(인)}
	};

	\node[anchor=center] at %
	(page cs:0.48+\apprvsigntwowidth/2-\apprvsignwidth-2mm,0.35){%
		%\pgfimage{example-image-a}
		\includegraphics[height=\apprvsigntwoheight]{\apprvsigntwo}
	};
	\node[anchor=east] at (page cs:0.535,0.35){%
		%\pgfimage{example-image-a}
		{\Large 심사위원\hspace{5mm}김 교 수\hspace{6mm}(인)}
	};
	
	\node[anchor=center] at %
	(page cs:0.48+\apprvsignthreewidth/2-\apprvsignwidth-2mm,0.35-14mm){%
		%\pgfimage{example-image-a}
		\includegraphics[height=\apprvsignthreeheight]{\apprvsignthree}
	};
	\node[anchor=east] at (page cs:0.535,0.35-14mm){%
		%\pgfimage{example-image-a}
		{\Large 심사위원\hspace{5mm}홍 판 서\hspace{6mm}(인)}
	};
	\end{tikzpicture}
	
	
	%\begin{myepigraph}
		%{\Large {몇년몇월며칠}}\\[11mm]
		%{\Large 심사위원장\hspace{5mm}박대감\hspace{6mm}(인)}\\[8mm]
		%{\Large 심사위원\hspace{5mm}김교수\hspace{6mm}(인)}\\[8mm]
		%{\Large 심사위원\hspace{5mm}홍판서\hspace{6mm}(인)}
	%\end{myepigraph}
	
	\vspace{20em}
	
	\begin{myepigraph}
	{\Large {몇년몇월며칠}}\\[11mm]
	{\Large 심사위원장\hspace{5mm}박 대 감\hspace{6mm}(인)}\\[8mm]
	{\Large 심사위원\hspace{5mm}김 교 수\hspace{6mm}(인)}\\[8mm]
	{\Large 심사위원\hspace{5mm}홍 판 서\hspace{6mm}(인)}
	\end{myepigraph}
	% 여기까지 원래 양식. 간격과 배치를 보기 위한 대조본으로 실제 코드의 
	% 커맨드들을 글자로 치환함.
\end{document}